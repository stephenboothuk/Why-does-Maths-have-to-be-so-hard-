\documentclass{SBBook}
\usepackage[T1]{fontenc}
\usepackage[utf8]{inputenc}
\usepackage{hyperref}
\usepackage{imakeidx}
\usepackage{pdfpages}
\usepackage[nobottomtitles]{titlesec}
\usepackage[font=scriptsize]{caption}
\usepackage{graphicx}
\usepackage{wrapfig}
\usepackage{amsmath, amsthm, amssymb, amsfonts}
\usepackage{mathtools}
\usepackage{harvard}
\usepackage{longtable}
\usepackage{pgfplots}
\usepackage{url}
\usepackage{lscape}
\usepackage[Conny]{fncychap}
%Options: Sonny, Lenny, Glenn, Conny, Rejne, Bjarne, Bjornstrup
\usepackage{minitoc}
%\usepackage{mtcoff}
\usepackage{./packages/SBooth}

\hypersetup{
pdftitle={Why Does Maths Have To Be So Hard? by Stephen Booth},
pdfsubject={Why Does Maths Have To Be So Hard?},
pdfauthor={Stephen Booth},
pdfkeywords={Math Maths Statistics Trigonometry Calculus Differentials Integrals},
colorlinks=true,
linkcolor=blue
}

\titleclass{\part}{top}
\titleformat{\part}[display]
  {\normalfont\huge\bfseries}{\centering\partname\  \thepart}{20pt}{\Huge\centering}
  
\pgfplotsset{compat=1.17}
\pgfmathdeclarefunction{gauss}{2}{%
  \pgfmathparse{1/(#2*sqrt(2*pi))*exp(-((x-#1)^2)/(2*#2^2))}%
}

\setcounter{secnumdepth}{5}


%\makeindex
\makeindex[name=terms, columns=2, title= Index of Terms, options= -s terms.ist]
\makeindex[name=people, columns=2, title = Notable People in Maths, options= -s terms.ist]

\newcommand{\ii}[1]{{\it #1}}
\newcommand{\bb}[1]{\textbf{#1}}
\begin{document}
\title{Why Does Maths Have To Be So Hard?}

\author{Stephen Booth MIET MBCS OLA \\ \url{https://orcid.org/0000-0003-3939-8493}}
%\author{}
\date{First Edition \\2022}
%\edition{First}
\maketitle
% \doparttoc
\dominitoc
%\part{Fore Matters}
\chapter[Foreword (and why I wrote this)]{Foreword \\ (and why I wrote this) \\ Not that anyone reads this bit of a book any way} \label{Foreword (and why I wrote this}
\label{Foreword}
\minitoc

\section*{Who this book is for}
Anyone who will find it helpful really.  A more helpful description might be to say that in writing this I'm presuming you're something like me as I started in on this.  It's been a few years, or maybe a few decades, since you last formally studied mathematics, you remember the bits you've had to use since then but the rest has faded into the mush of memory.  Now --- maybe because you have to help your kids with their homework, for your job or just because you're interested --- you want to brush up on your maths skills and remind yourself of the things you used to know, and maybe learn a few new things along the way.

All that said, if you're a school pupil or university student looking for a supplementary text to back up your main learning and think this might help then, please, enjoy and I hope you get some value from it.  Similarly, if you are a maths teacher who wants to use this as a supplementary text for your classes then go ahead, if you think I've got something wrong in the maths then please tell me at \href{mailto:stephenbooth.uk@gmail.com}{stephenbooth.uk@gmail.com} and I'll look into correcting in later editions; to err is human but to really foul things up requires a human sat at a computer, and I'm a human sat at a computer writing this (in \LaTeXe{} if that means anything to you).

\subsection*{What won't be covered here, or what I'm going to presume you already know}
I don't propose to cover basic numeracy skills such as basic arithmetic (adding, subtracting, dividing and multiplying).  If I feel the need and work out a way to teach them from a book then I might include them in in some later edition.  I will be covering order of operations and types of numbers as they are something I know confuse a lot of people.

I also won't be covering how to use a calculator.  I will presume that you have access to a scientific calculator and can read the manual to find out how it works.  There are simply too many different makes and models in the world, plus calculator apps on smart phones and tablets, for me to cover even a fraction of whats out there.  Personally I tend to use the \texttt{CASIO fx-83GT Plus}, \texttt{CASIO fx-570 ES Plus} or \texttt{CASIO fx-991EX}.  This is not an endorsement of the brand, they just happened to be what was available with the functions I needed at the right price when I needed one.  Also I have used CASIO scientific calculators since secondary school so, as they have kept the interface fairly standard, a lot of the keystrokes are in my muscle memory.  It shouldn't matter what make or model you use so long as it has the usual functions and you know how to use it (read the manual if need be).  There is a caveat when we get to order of operations, but I'll explain it then.

I won't be covering slide rules.  I know two people in the world who still use slide rules and in both our cases it's more for nostalgia reasons than anything.  If you want to learn the slide rule then I suggest getting a \textit{Faber Castell Student 10 inch Reitz} slide rule and \textit{``A Manual of the Slide Rule''} by J E Thompson\index[people]{Thompson, J E} (\textsc{ISBN: 9-781479-444151}), both of which can be found from a number of online sellers.  The book is pretty much the definitive work on the slide rule and has been around since 1930, the slide rule is the one used in many of the exercises of the book so, whilst any slide rule should work, there will be familiarity with the book.

\section*{Maths?  Don't you mean Math?}
I, Stephen Booth, author of this monstrosity, am British where the subject of Mathematics is abbreviated to Maths.  You may not be British and the custom in your country may be to say Math.  Similarly you may drop the `u' from colour and other such differences.  Please forgive my use of the language with which I am familiar, International English, rather than United States English.

Also my spell checker puts a red squiggly underline if I miss off the `s' and i make enough spelling mistakes anyway that I need that to see where I have misspelled words without it getting lost in a mire of words that are correct in another dictionary.

\section*{\textbf{This book is FREE}}
This book is provided free for anyone to use for self study or as a supplmentary text for teaching.  If someone has charged you for this book, other than reasonable printing costs, they are are in breach of license for this book.

This book is published under a creative commons Non-Commercial, No Derivatives, Attribution license (\href{https://creativecommons.org/licenses/by-nc-nd/4.0/}{CC BY-NC-ND 4.0})\index[terms]{Creative Commons}.  That is it must be provided as-is, without editing and at no cost (beyond reasonable printing cost if printed). I do encourage digital copies to be used, unless that is impractical for your situation, both to reduce the environmental impact and because it includes clickable links that take you to related points in the text and to relevant external resources (for example the applicable Creative Commons license above and in the next paragraph to my Ko-Fi page).

If you want to send me a tip or donation then you can do so via  \href{https://ko-fi.com/stephenbooth}{Ko-Fi}

\section*{Thanks}
Plagiarism, it is said, is to steal from one source, research from many sources.  In that case this is a very well researched book.  I have dipped into too many sources to name all authors and content creators whose work shaped in some way or provided information for this book.  I would like to particularly thank YouTubers \href{https://www.youtube.com/c/misterwootube}{Eddie Woo}\index[people]{Woo, Eddie}, \href{https://www.youtube.com/c/DrTreforBazett}{Dr. Trefor Bazett}\index[people]{Bazett, Trefor (Dr)} and \href{https://www.youtube.com/c/TheMathSorcerer}{The Math Sorcerer}\index[people]{The Math Sorcerer} for inspiration and ideas to write this even if they didn't know they were doing it.  Please follow them.  I'd also like to recognise Ravi Subramanian for being the only other person I know who still uses a slide rule.

\tableofcontents
\listoffigures
\listoftables

\part{Background}
% \parttoc
\chapter{Introduction}

\minitoc

\section{Actually, why \emph{does} maths have to be so hard}
Maths, unlike many other subjects you might learn in school, very much builds upon itself.  Topics don't exist in isolation, what you learn today builds on what you learned yesterday, or last week and maybe something you learned a few months ago.  To a degree that is true of any STEM\index[terms]{STEM} (Science Technology Engineering Maths) subject.

That's not the case in many other subjects, in Geography knowing the principle exports of South Africa won't help you with understanding Terminal Moraines in the post-glacial landscapes of Northern Europe, similarly in History whilst some knowledge of the later Plantagenet royals and the doings of the Tudor dynasty may give you some context for the Stewarts and the English Civil War, but it's by no means required.

In maths, however, each topic you learn provides a foundation for future learning.  For example, once you have learned the Number System\index[terms]{Number Systems}, Systematic Problem Solving\index[terms]{Systematic Problem Solving}\index[terms]{Problem Solving, Systematic} and Arithmetic\index[terms]{Arithmetic} you have the basis to understand Basic Algebra\index[terms]{Basic Algebra}\index[terms]{Algebra!Basic}, once you understand Basic Algebra that gives you the tools you need to learn Geometry\index[terms]{Geometry} and more advanced Algebra\index[terms]{Advanced Algebra}\index[terms]{Algebra!Advanced}.  Geometry and Algebra give you the tools you need to learn Trigonometry and from there you can learn Calculus\index[terms]{Calculus}.  Geometry and Algebra also give you the tools you need to learn Logic which opens up Statistics\index[terms]{Statistics}, Topology\index[terms]{Topology}, Probability\index[terms]{Probability} and Abstract Algebra\index[terms]{Abstract Algebra}\index[terms]{Algebra!Abstract}.

This is both the strength and weakness of learning maths.  Each topic follows on from the topics that went before it so, so long as you are aware of this and have studied the previous topics, the learning builds like a wall with the foundations stones (topics) supporting the stones laid atop them.  Unfortunately, if you missed or misunderstood  an earlier topic you may find yourself unable to learn later topics because you haven't been given the tools you need, and may not even be away that the tools exist.  Some foundation stones are missing from your wall of maths so the later stones are unstable at best and completely unsupported at all at worst.  There are all sort of reasons that you may be missing or misunderstanding the earlier topics.  Perhaps you missed some school and weren't able to catch up?  Perhaps your teacher just didn't know how to explain them in a way that you could understand, or didn't have time in their class schedule to address your learning style?  Perhaps you changed schools and your new school had already covered them when you joined and you didn't have a chance to see what you missed and fill in the gaps?

It's quite common for adults who struggle with maths to report that it made sense up to a point, then it stopped making sense.  Often the point where it stopped making sense coincided with a change in their life such as changing schools, a parental divorce or separation, a period of sickness or hospitalisation, or a major bereavement.  In my case the school I had been attending closed down due to bankruptcy meaning that I missed about 6 months of school during which my parents' divorce was finalised which involved a lot of fighting and each trying to get me to side with them against the other so I was under a lot of stress and struggled to concentrate on self study.  When I did start a new school the maths teachers were indifferent to say the least, looking back I think they were definitely suffering from burnout due to reductions in budgets leading to reductions in teaching staff but increases in class sizes.  In a two year period, average class size had risen from 23 to 35 I remember one complaining, which was kind of relevant as the class was on the different types of averages.

\section{What I presume you already know}
\subsection{Arithmetic}
The main thing is, I'm going to presume that you already know basic arithmetic (adding, subtracting, multiplying and dividing) to at least a level that you can add up your shopping bill, adjust a recipe for the number of people eating and work out how much you are saving from a discount.  If you struggle with this, and are not currently in education, then I encourage you to check with your local colleges and government to see what adult numeracy courses are available.  If you are a member of a trade union then talk to your local representative as many will run courses for members.  This is something that does need to be covered face-to-face by a professional educator, in part because there are a few disabilities such as Dysorthographia\index[terms]{Dysorthographia}\index[terms]{Neurodiversity!Dysorthographia} and Dyscalculia\index[terms]{Dyscalculia}\index[terms]{Neurodiversity!Dyscalculia} that can make it very hard for people to handle arithmetic and a professional educator may be able to spot these and help you find ways to work around them if needed (I have Dyspraxia\index[terms]{Dyspraxia}\index[terms]{Neurodiversity!Dyspraxia} with causes diffculty with the organisation of thought and movement, which translates to I'm clumsy and I literally don't think the way other people, Neurotypical\index[terms]{Neurotypical}, do, this latter part impacted on my ability to learn maths as it makes it hard for me to keep intermediate figures like subtotals in my head so if, for example, I need to workout the result of one formula, then workout the result of another formula if I don't write the first one down I'll have forgotten it by the time I've solved the second one, similarly if I need to look up a figure (say a constant or a logarithm\index[terms]{Logarithm}) then us it in a calculation that can take a few attempts to get it from the lookup table in a book to the calculation if I don't have some paper and a pencil).  Or it could be that you just had a poor teacher and it was never caught through your schooling.  Either way it is nothing to be embarrassed about at all, I've known some very smart people (including two maths professors) who couldn't split a restaurant bill without a calculator.

\subsection{Using a calculator}
A calculator will be helpful in some of the exercises.  You should have and have learned how to use the functions of a scientific calculator; that is a calculator that has keys for functions such as Sin, Cos, Tan, $X^2$, $\sqrt{x}$, $log_10$, Ln \&c, and modes for displaying numbers to a certain number of decimal places (usually referred as FIX) or Scientific format (usually SCI) and for representing angles in degrees (DEG) or radians (RAD, more on those later).  Unfortunately, there are so many variations of makes and models, and physical calculators vs apps on computers, smartphones and tablets, that it would be impossible to cover even a small proportion of those available.  Your calculator most likely came with a manual or maybe a link to where you can download one, I recommend keeping it to hand and checking it any time you find something you don't recognise.

\subsection{Internet Browsing}
This book is distributed via the internet as a PDF (Portable Document Format) file so I presume that you have access to and reasonable familiarity with browsing the internet.  If this is something you struggle with then please check with local colleges in your area and your local government as many run free or very cheap courses to teach basic internet and digital skills.


\part{Numbers}
% \parttoc
\chapter{Types of Number}\label{typesofnumber}

\minitoc

\section{Natural Numbers}\label{naturalnumbers}
\index[terms]{Natural Numbers}\index[terms]{Numbers!Natural}\index[terms]{Counting Numbers}\index[terms]{Numbers!Counting}

\subsection{Prime Numbers}\label{primenumbers}
\index[terms]{Numbers!Prime}\index[terms]{Prime Numbers}

\subsection{Composite Numbers}\label{compositenumbers}
\index[terms]{Numbers!Composite}\index[terms]{Composite Numbers}

\section{Integers}\label{integers}
\index[terms]{Numbers!Integers}\index[terms]{Integers}

\section{Real Numbers}\label{realnumbers}
\index[terms]{Numbers!Real}\index[terms]{Real Numbers}

\subsection{Rational Numbers}\label{rationalnumbers}
\index[terms]{Numbers!Rational}\index[terms]{Rational Numbers}

\subsection{Irrational Numbers}\label{irrationalnumbers}
\index[terms]{Numbers!Irrational}\index[terms]{Irrational Numbers}

\section{Complex Numbers}\label{complexnumbers}
\index[terms]{Numbers!Complex}\index[terms]{Complex Numbers}



\chapter{Bases}
\minitoc
\section{What are bases?}\index[terms]{Bases}

\section{Base 10 (Decimal)}\index[terms]{Base!10}\index[terms]{Base!Decimal}

\section{Base 2 (Binary)}
\index[terms]{Base!2}\index[terms]{Base!Binary}\index[terms]{Binary}

\section{Base 3 (Trinary)}\index[terms]{Base!3}\index[terms]{Base!Trinary}

\section{Base 8 (Octal)}\index[terms]{Base!8}\index[terms]{Base!Octal}\index[terms]{Octal}

\section{Base 16 (Hexadecimal)}\index[terms]{Base!16}\index[terms]{Base!Hexadecimal}\index[terms]{Hexadecimal}

\part{Algebra}
\input{chapters/basicalgebra.tex}

\part{Shapes}

\input{chapters/trigonometry.tex}

\part{Calculus}
\input{chapters/calculus.tex}

\part{Appendices} \label{Appendices}
% \parttoc
\appendix
\chapter{Answers to exercises} \label{answerstoexercises}
\minitoc
Here are the answers to various exercises contained in the text.
\section{Introduction}

\section{Types of Number}

\section{Bases}
\cleardoublepage
\printindex[terms]

\cleardoublepage
\printindex[people]
\part{End Matter} \label{End Matter}

This book was typeset, probably quite poorly, in \LaTeXe by the author as he tried to refresh his knowledge of maths and figured that he may as well write it down and see if anyone else might be interested.

The information in this document have been compiled from a wide variety of sources, as the saying goes ``To steal from one source is plagiarism, to steal from many is research'', in which case this is a very well researched work.  

Updates, additions, changes and corrections may be made as and when deemed necessary.

The rights of Stephen Booth to be identified as the author of this work are hereby asserted. This imprint \copyright {\the\year}  Stephen Booth.

This book is published under a creative commons Non-Commercial, No Derivatives, Attribution license (\href{https://creativecommons.org/licenses/by-nc-nd/4.0/}{CC BY-NC-ND 4.0}).  That is it must be provided as-is, without editing and at no cost (beyond reasonable printing cost if printed).
\end{document}
