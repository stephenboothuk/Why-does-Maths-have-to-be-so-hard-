\chapter{Introduction}

\minitoc

\section{Actually, why \emph{does} maths have to be so hard}
Maths, unlike many other subjects you might learn in school, very much builds upon itself.  Topics don't exist in isolation, what you learn today builds on what you learned yesterday, or last week and maybe something you learned a few months ago.  To a degree that is true of any STEM\index[terms]{STEM} (Science Technology Engineering Maths) subject.

That's not the case in many other subjects, in Geography knowing the principle exports of South Africa won't help you with understanding Terminal Moraines in the post-glacial landscapes of Northern Europe, similarly in History whilst some knowledge of the later Plantagenet royals and the doings of the Tudor dynasty may give you some context for the Stewarts and the English Civil War, but it's by no means required.

In maths, however, each topic you learn provides a foundation for future learning.  For example, once you have learned the Number System\index[terms]{Number Systems}, Systematic Problem Solving\index[terms]{Systematic Problem Solving}\index[terms]{Problem Solving, Systematic} and Arithmetic\index[terms]{Arithmetic} you have the basis to understand Basic Algebra\index[terms]{Basic Algebra}\index[terms]{Algebra!Basic}, once you understand Basic Algebra that gives you the tools you need to learn Geometry\index[terms]{Geometry} and more advanced Algebra\index[terms]{Advanced Algebra}\index[terms]{Algebra!Advanced}.  Geometry and Algebra give you the tools you need to learn Trigonometry and from there you can learn Calculus\index[terms]{Calculus}.  Geometry and Algebra also give you the tools you need to learn Logic which opens up Statistics\index[terms]{Statistics}, Topology\index[terms]{Topology}, Probability\index[terms]{Probability} and Abstract Algebra\index[terms]{Abstract Algebra}\index[terms]{Algebra!Abstract}.

This is both the strength and weakness of learning maths.  Each topic follows on from the topics that went before it so, so long as you are aware of this and have studied the previous topics, the learning builds like a wall with the foundations stones (topics) supporting the stones laid atop them.  Unfortunately, if you missed or misunderstood  an earlier topic you may find yourself unable to learn later topics because you haven't been given the tools you need, and may not even be away that the tools exist.  Some foundation stones are missing from your wall of maths so the later stones are unstable at best and completely unsupported at all at worst.  There are all sort of reasons that you may be missing or misunderstanding the earlier topics.  Perhaps you missed some school and weren't able to catch up?  Perhaps your teacher just didn't know how to explain them in a way that you could understand, or didn't have time in their class schedule to address your learning style?  Perhaps you changed schools and your new school had already covered them when you joined and you didn't have a chance to see what you missed and fill in the gaps?

It's quite common for adults who struggle with maths to report that it made sense up to a point, then it stopped making sense.  Often the point where it stopped making sense coincided with a change in their life such as changing schools, a parental divorce or separation, a period of sickness or hospitalisation, or a major bereavement.  In my case the school I had been attending closed down due to bankruptcy meaning that I missed about 6 months of school during which my parents' divorce was finalised which involved a lot of fighting and each trying to get me to side with them against the other so I was under a lot of stress and struggled to concentrate on self study.  When I did start a new school the maths teachers were indifferent to say the least, looking back I think they were definitely suffering from burnout due to reductions in budgets leading to reductions in teaching staff but increases in class sizes.  In a two year period, average class size had risen from 23 to 35 I remember one complaining, which was kind of relevant as the class was on the different types of averages.

\section{What I presume you already know}
\subsection{Arithmetic}
The main thing is, I'm going to presume that you already know basic arithmetic (adding, subtracting, multiplying and dividing) to at least a level that you can add up your shopping bill, adjust a recipe for the number of people eating and work out how much you are saving from a discount.  If you struggle with this, and are not currently in education, then I encourage you to check with your local colleges and government to see what adult numeracy courses are available.  If you are a member of a trade union then talk to your local representative as many will run courses for members.  This is something that does need to be covered face-to-face by a professional educator, in part because there are a few disabilities such as Dysorthographia\index[terms]{Dysorthographia}\index[terms]{Neurodiversity!Dysorthographia} and Dyscalculia\index[terms]{Dyscalculia}\index[terms]{Neurodiversity!Dyscalculia} that can make it very hard for people to handle arithmetic and a professional educator may be able to spot these and help you find ways to work around them if needed (I have Dyspraxia\index[terms]{Dyspraxia}\index[terms]{Neurodiversity!Dyspraxia} with causes diffculty with the organisation of thought and movement, which translates to I'm clumsy and I literally don't think the way other people, Neurotypical\index[terms]{Neurotypical}, do, this latter part impacted on my ability to learn maths as it makes it hard for me to keep intermediate figures like subtotals in my head so if, for example, I need to workout the result of one formula, then workout the result of another formula if I don't write the first one down I'll have forgotten it by the time I've solved the second one, similarly if I need to look up a figure (say a constant or a logarithm\index[terms]{Logarithm}) then us it in a calculation that can take a few attempts to get it from the lookup table in a book to the calculation if I don't have some paper and a pencil).  Or it could be that you just had a poor teacher and it was never caught through your schooling.  Either way it is nothing to be embarrassed about at all, I've known some very smart people (including two maths professors) who couldn't split a restaurant bill without a calculator.

\subsection{Using a calculator}
A calculator will be helpful in some of the exercises.  You should have and have learned how to use the functions of a scientific calculator; that is a calculator that has keys for functions such as Sin, Cos, Tan, $X^2$, $\sqrt{x}$, $log_10$, Ln \&c, and modes for displaying numbers to a certain number of decimal places (usually referred as FIX) or Scientific format (usually SCI) and for representing angles in degrees (DEG) or radians (RAD, more on those later).  Unfortunately, there are so many variations of makes and models, and physical calculators vs apps on computers, smartphones and tablets, that it would be impossible to cover even a small proportion of those available.  Your calculator most likely came with a manual or maybe a link to where you can download one, I recommend keeping it to hand and checking it any time you find something you don't recognise.

\subsection{Internet Browsing}
This book is distributed via the internet as a PDF (Portable Document Format) file so I presume that you have access to and reasonable familiarity with browsing the internet.  If this is something you struggle with then please check with local colleges in your area and your local government as many run free or very cheap courses to teach basic internet and digital skills.
