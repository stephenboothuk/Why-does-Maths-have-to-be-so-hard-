\chapter[Foreword (and why I wrote this)]{Foreword \\ (and why I wrote this) \\ Not that anyone reads this bit of a book any way} \label{Foreword (and why I wrote this}
\label{Foreword}
\minitoc

\section*{Who this book is for}
Anyone who will find it helpful really.  A more helpful description might be to say that in writing this I'm presuming you're something like me as I started in on this.  It's been a few years, or maybe a few decades, since you last formally studied mathematics, you remember the bits you've had to use since then but the rest has faded into the mush of memory.  Now --- maybe because you have to help your kids with their homework, for your job or just because you're interested --- you want to brush up on your maths skills and remind yourself of the things you used to know, and maybe learn a few new things along the way.

All that said, if you're a school pupil or university student looking for a supplementary text to back up your main learning and think this might help then, please, enjoy and I hope you get some value from it.  Similarly, if you are a maths teacher who wants to use this as a supplementary text for your classes then go ahead, if you think I've got something wrong in the maths then please tell me at \href{mailto:stephenbooth.uk@gmail.com}{stephenbooth.uk@gmail.com} and I'll look into correcting in later editions; to err is human but to really foul things up requires a human sat at a computer, and I'm a human sat at a computer writing this (in \LaTeXe{} if that means anything to you).

\subsection*{What won't be covered here, or what I'm going to presume you already know}
I don't propose to cover basic numeracy skills such as basic arithmetic (adding, subtracting, dividing and multiplying).  If I feel the need and work out a way to teach them from a book then I might include them in in some later edition.  I will be covering order of operations and types of numbers as they are something I know confuse a lot of people.

I also won't be covering how to use a calculator.  I will presume that you have access to a scientific calculator and can read the manual to find out how it works.  There are simply too many different makes and models in the world, plus calculator apps on smart phones and tablets, for me to cover even a fraction of whats out there.  Personally I tend to use the \texttt{CASIO fx-83GT Plus}, \texttt{CASIO fx-570 ES Plus} or \texttt{CASIO fx-991EX}.  This is not an endorsement of the brand, they just happened to be what was available with the functions I needed at the right price when I needed one.  Also I have used CASIO scientific calculators since secondary school so, as they have kept the interface fairly standard, a lot of the keystrokes are in my muscle memory.  It shouldn't matter what make or model you use so long as it has the usual functions and you know how to use it (read the manual if need be).  There is a caveat when we get to order of operations, but I'll explain it then.

I won't be covering slide rules.  I know two people in the world who still use slide rules and in both our cases it's more for nostalgia reasons than anything.  If you want to learn the slide rule then I suggest getting a \textit{Faber Castell Student 10 inch Reitz} slide rule and \textit{``A Manual of the Slide Rule''} by J E Thompson\index[people]{Thompson, J E} (\textsc{ISBN: 9-781479-444151}), both of which can be found from a number of online sellers.  The book is pretty much the definitive work on the slide rule and has been around since 1930, the slide rule is the one used in many of the exercises of the book so, whilst any slide rule should work, there will be familiarity with the book.

\section*{Maths?  Don't you mean Math?}
I, Stephen Booth, author of this monstrosity, am British where the subject of Mathematics is abbreviated to Maths.  You may not be British and the custom in your country may be to say Math.  Similarly you may drop the `u' from colour and other such differences.  Please forgive my use of the language with which I am familiar, International English, rather than United States English.

Also my spell checker puts a red squiggly underline if I miss off the `s' and i make enough spelling mistakes anyway that I need that to see where I have misspelled words without it getting lost in a mire of words that are correct in another dictionary.

\section*{\textbf{This book is FREE}}
This book is provided free for anyone to use for self study or as a supplmentary text for teaching.  If someone has charged you for this book, other than reasonable printing costs, they are are in breach of license for this book.

This book is published under a creative commons Non-Commercial, No Derivatives, Attribution license (\href{https://creativecommons.org/licenses/by-nc-nd/4.0/}{CC BY-NC-ND 4.0})\index[terms]{Creative Commons}.  That is it must be provided as-is, without editing and at no cost (beyond reasonable printing cost if printed). I do encourage digital copies to be used, unless that is impractical for your situation, both to reduce the environmental impact and because it includes clickable links that take you to related points in the text and to relevant external resources (for example the applicable Creative Commons license above and in the next paragraph to my Ko-Fi page).

If you want to send me a tip or donation then you can do so via  \href{https://ko-fi.com/stephenbooth}{Ko-Fi}

\section*{Thanks}
Plagiarism, it is said, is to steal from one source, research from many sources.  In that case this is a very well researched book.  I have dipped into too many sources to name all authors and content creators whose work shaped in some way or provided information for this book.  I would like to particularly thank YouTubers \href{https://www.youtube.com/c/misterwootube}{Eddie Woo}\index[people]{Woo, Eddie}, \href{https://www.youtube.com/c/DrTreforBazett}{Dr. Trefor Bazett}\index[people]{Bazett, Trefor (Dr)} and \href{https://www.youtube.com/c/TheMathSorcerer}{The Math Sorcerer}\index[people]{The Math Sorcerer} for inspiration and ideas to write this even if they didn't know they were doing it.  Please follow them.  I'd also like to recognise Ravi Subramanian for being the only other person I know who still uses a slide rule.